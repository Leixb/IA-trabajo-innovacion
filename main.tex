% vim: spell spelllang=es:
\documentclass[12pt, oneside]{article}
\usepackage[a4paper, left=2.5cm, right=2.5cm, top=2.5cm, bottom=2.5cm]{geometry}

\usepackage[utf8]{inputenc} % Use unicode
\usepackage[T1]{fontenc}
\usepackage[spanish,es-tabla]{babel} % Names in spanish


\usepackage{xcolor}
\usepackage{siunitx}

%% Bibliography:
%\usepackage{comment}
%\usepackage[
    %backend=biber,
    %style=numeric,
%]{biblatex}
%\DeclareNameAlias{default}{last-first}

%\usepackage{csquotes}       % For bibliography quotations
%\DeclareQuoteAlias{spanish}{catalan}

%\addbibresource{biblio.bib}
%% see:
%% https://www.sharelatex.com/learn/Bibliography_management_in_LaTeX#The_bibliography_file

%\usepackage{datetime} % Customize date
%% \monthyeardate\today gives the date without the day
%\newdateformat{monthyeardate}{%
    %\monthname[\THEMONTH], \THEYEAR}

% For cross references
\PassOptionsToPackage{hyphens}{url}\usepackage[colorlinks = true]{hyperref}
\usepackage[catalan]{varioref}
%\usepackage{cleveref}
%hyperref configuration so that it doesn't contrast so much colorlinks,
\hypersetup{
   linkcolor={black},
   citecolor={black},
   %linkcolor={red!50!black},
   %citecolor={blue!50!black},
   urlcolor={blue!80!black}
}

\usepackage{mathtools}  % amsmath + more
\usepackage{amsthm}     % Theorem enviroment
\usepackage{amssymb}    % More symbols
\usepackage{amstext}    % Text inside mathenv

\usepackage{relsize}    % Bigger math with mathlarger{___}
\usepackage{nicefrac}   % nice fractions in one line

\usepackage[export]{adjustbox}  % Adjust table size
\usepackage{float}              % Force tables and images position (H and H!)
\usepackage{wrapfig}            % Wrap images like in HTML

\usepackage{tabularx, colortbl, booktabs}    % Better tables
\usepackage{longtable}                      % Multiple page table

% Split cell in lines and more formating options inside table
\usepackage{array, multirow, multicol, makecell}

%\usepackage{subcaption}                     % Subfigures
%\usepackage[framemethod=tikz]{mdframed}     % Custom frames

%\usepackage[bottom]{footmisc} % Footnotes at bottom of page

%\usepackage[alsoload=hep]{siunitx}          % SI units and uncertainties
%\sisetup{locale = FR}                       % Commas and so on for spanish
%\sisetup{separate-uncertainty=true}
%\sisetup{
  %per-mode=fraction,
  %fraction-function=\nicefrac
%}

%\usepackage{tikz}
%%\usetikzlibrary{arrows}
%%\usetikzlibrary{scopes}
%\usetikzlibrary{babel}

%\usepackage{listings}       % For code blocks

%% Custom code highlight
%\definecolor{codegreen}{rgb}{0,0.6,0}
%\definecolor{codegray}{rgb}{0.5,0.5,0.5}
%\definecolor{codepurple}{rgb}{0.58,0,0.82}
%\definecolor{backcolour}{rgb}{0.95,0.95,0.92}
%\definecolor{lightblue}{RGB}{135,206,250}

%\lstdefinestyle{mystyle}{ backgroundcolor=\color{backcolour},
    %commentstyle=\color{codegreen}, keywordstyle=\color{blue},
    %numberstyle=\tiny\color{codegray}, stringstyle=\color{red},
    %identifierstyle=\color{black}, basicstyle=\footnotesize,
    %%breakatwhitespace=false,
    %breaklines=true,
    %%captionpos=b,                    keepspaces=true,
    %numbers=left,                    numbersep=5pt,
    %showspaces=false,
    %%showstringspaces=false, showtabs=false,
    %tabsize=4
%}
%\lstset{style=mystyle}

\newcommand{\whitepage}{
    \clearpage\thispagestyle{empty}\addtocounter{page}{-1} \newpage \clearpage
}

% Add command before appendix session for page numbering: A-1
%\newcommand{\appendixpagenumbering}{
    %\break
    %\pagenumbering{arabic}
    %\renewcommand{\thepage}{\thesection-\arabic{page}}
%}

%% Custom Math operators (functions not in italic in math mode):
%\DeclareMathOperator{\arcsec}{arcsec}
%\DeclareMathOperator{\arccot}{arccot}
%\DeclareMathOperator{\arccsc}{arccsc}
%\DeclareMathOperator{\cis}{cis}


\usepackage[bottom]{footmisc}

\usepackage{amsmath}
\usepackage[justification=centering]{caption}

\title{IA Búsqueda local}
\author{%
    Aleix Boné\\
    Alex Herrero\\
    Moisés Balcells
}
\date{%
Marzo 2020
}

\begin{document} 


\thispagestyle{empty}
\clearpage
\setcounter{page}{-1}

\begin{titlepage}
{
    \centering
    \null
    \vfill
    {\Large Inteligencia Artificial\par}
    \vspace{2em}
    {\Huge \bfseries 
    Práctica de búsqueda local
    \par}
    \vspace{2em}
    {\large \scshape 
    Marzo 2020
    \par}
    \vfill
\begin{center}
    
\end{center}
    \vspace{3cm}

    \vfill
    {\raggedleft \large
Aleix Boné Ribó\\
Alex Herrero Pons\\
    Moisés Balcells
        \par}
}
\end{titlepage}

\pagebreak

\thispagestyle{empty}
\clearpage
\setcounter{page}{0}

\tableofcontents

\pagebreak

\section{Trabajo de innovación}

El tema sobre el que trabajaremos es el escaneo inteligente con deep learning aplicado a la resonancia magnética. 
En este ámbito la inteligencia artificial se utiliza para hacer el escaneo de imágenes más preciso y rápido. 
En concreto nos centraremos en el producto AIRx™ de general electric healthcare.



La búsqueda de la información del trabajo la hemos repartido de la siguiente manera:
\begin{itemize}
\item Moisés
    \begin{itemize}
\item búsqueda de información para la descripción del producto o servicio.
\item búsqueda de información para la descripción de las técnicas de IA que se han utilizado.
    \end{itemize}
\item Aleix
    \begin{itemize}
\item búsqueda de información para la descripción de cómo han sido utilizadas las técnicas.
\item búsqueda de información para la explicación de porqué es un producto/servicio innovador y la naturaleza de la Innovación (innovación en la técnica/método de IA, uso innovador de técnicas ya existentes).
    \end{itemize}
\item Alex
    \begin{itemize}
\item búsqueda de información para el impacto del producto en la empresa (beneficios, riesgos, posición en el mercado)
\item búsqueda de información para el impacto del producto en el usuario o en la sociedad (beneficios y riesgos)
    \end{itemize}
\item Todos:
    \begin{itemize}
\item Recopilación de la bibliografía y referencias utilizadas para la elaboración del documento.
    \end{itemize}
\end{itemize}

\subsection{Referencias relevantes:}

\begin{itemize}
\item \url{https://blog.tensorflow.org/2019/03/intelligent-scanning-using-deep-learning.html?m=1}:
(Cons. 11/03)
Descripción del producto, innovación del producto técnicas de IA utilizadas, como se han utilizado las técnicas, impacto en la empresa
\item \url{https://www.gehealthcare.com/article/magnetic-resonance-imaging-using-ai-a-new-deep-learning-tool}:
(Cons. 11/03)
Descripción del producto, innovación del producto, técnicas de IA utilizadas, impacto en el usuario o la sociedad
\item \url{https://www.gehealthcare.com/article/how-advanced-applications-and-ai-are-impacting-mri}:
(Cons. 11/03)
Técnicas de IA utilizadas, impacto en el usuario o la sociedad
\item \url{http://www.gesignapulse.com/signapulse/spring_2019/MobilePagedArticle.action?articleId=1488815&app=false#articleId1488815}:
(Cons. 14/03)
Técnicas de IA utilizadas, como se han utilizado las técnicas, impacto en el usuario o en la empresa
\item \url{http://www.gesignapulse.com/signapulse/spring_2018/MobilePagedArticle.action?articleId=1396203&app=false#articleId1396203}:
(Cons. 15/03)
Descripción del producto, innovación del producto, impacto del producto en el usuario o en la sociedad, impacto del producto en la empresa
\item \url{https://www.auntminnie.com/index.aspx?sec=rca&sub=ecr_2019&pag=dis&ItemID=124750}:
(Cons. 14/03)
técnicas IA utilizadas, impacto del producto en el usuario o en la sociedad, impacto del producto en la empresa
\item \url{https://www.dotmed.com/news/story/46576}:
(Cons. 14/03)
Descripción del producto, innovación del producto, impacto del producto en el usuario o en la sociedad, impacto del producto en la empresa
\item \url{http://www.gesignapulse.com/signapulse/autumn_2018/MobilePagedArticle.action?articleId=1444512&app=false#articleId1444512}:
(Cons. 14/03)
Descripción del producto,innovación del producto,  técnicas de IA utilizadas
\item \url{https://medium.com/stanford-ai-for-healthcare/dont-just-scan-this-deep-learning-techniques-for-mri-52610e9b7a85}:
(Cons. 16/03)
Descripción del producto, innovación del producto,  técnicas de IA utilizadas, como se han utilizado las técnicas
\item \url{https://www.intel.ai/solutions/gehc-airx/}:
(Cons. 22/03)
Descripción de las diferentes CNNs que se usan.
\end{itemize}

La principal dificultad que hemos encontrado al buscar información ha sido que no dominamos la terminología
de los escáneres MRI ni su funcionamiento, por lo que a parte de buscar información sobre el producto y la
tecnología nos hemos tenido que informar también sobre todo el proceso detrás de una sesión de MRI y sus
complejidades.

\end{document}